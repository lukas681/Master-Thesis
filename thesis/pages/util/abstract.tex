\chapter{Abstract}

\begin{abstract}
{\sffamily
For a graph \G, a set $D$ is called a \textit{semitotal dominating set}, if $D$ is a dominating set and every vertex $v \in D$ is within distance two to another witness $v' \in D$. 
The \msdom problem is to find a semitotal dominating set of minimum cardinality. 
The semitotal domination number $\gamma_{t2}(G)$ is the minimum cardinality of a semitotal dominating set and is squeezed between the domination number $\gamma(G)$ and the total domination number $\gamma_t(G)$.
 Given \G and a positive integer $k$, the \sdomd problem asks if $\gamma_{t2} \leq k$.

After introduced by Goddard, Henning and McPillan~\cite{Goddard2014}, \NP-completeness of the problem was shown for various graph classes like general graphs, \emph{split}, \emph{planar}, \emph{chordal bipartite} and \emph{circle} graphs~\cite{Henning2019, Kloks2021}.
Contrary, there exist polynomial-time algorithms for \emph{block} and \emph{interval} graphs as well as for \emph{graphs of bounded mim-width}, \emph{graphs of bounded clique-width}~\cite{Kloks2021, Galby2020,Courcelle1990,Henning2022,Henning2019}.
After giving a status about the complexity of the problem, we start a systematic look through the lens of \textit{parameterized complexity} by showing that \sdom is $\WTWOhs$-hard for bipartite graphs and split graphs when parameterized by solution size.
On the positive side, we extend a technique proposed by Alber et al.~\cite{Alber2004} for \dom to construct a linear kernel of size $\kernelsize \cdot k$ for \sdom on planar graphs.
 
 This result complements known linear kernels for other domination problems like \cdom, \rbdom, \efdom, \eddom, \idom, and \dirdom on planar graphs \cite{Diekert2005,Garnero2018,Guo2007,Garnero2017,Luo2013,Alber2006}

\textbf{Keywords: }Domination; Semitotal Domination; Parameterized Complexity; Planar Graphs; Linear Kernel; Problem Reduction; Graph Theory
} 
\end{abstract}

\newpage

\begin{abstract}
{\sffamily

Ein Graph \G und eine Menge $D$ wird als \textit{halbtotale stabile Menge} bezeichnet, falls $D$ eine stabile Menge ist und jeder Knoten $v \in D$ maximal einen Abstand von zwei zu einem anderen Zeugen $v' \in D$ besitzt. 
Das \msdomDE Problem frägt nach einer halbtotalen stabilen Menge von minimaler Kardinalität. 
Sei $\gamma_{t2}(G)$ die minimale Kardinalität einer halbtotalen stabilen Menge. Diese ist zwischen der minimalen stabilen Menge $\gamma(G)$ and der minimalen total stabilen Menge $\gamma_t(G)$ eingezwängt.
Gegeben \G und ein positives $k$, das \sdomDE Problem frägt, ob $\gamma_{t2} \leq k $ ist.

Nachdem das Problem von Goddard, Henning und McPillan~\cite{Goddard2014} eingeführt wurde, konnte \NP-vollständigkeit für viele Graphklassen wie \emph{split}, \emph{plättbare}, \emph{chordal bipartite} und \emph{zirkuläre} Graphen bereits gezeigt werden~\cite{Henning2019, Kloks2021}.
Andererseits existieren polynomialzeit Algorithmen sowohl für \emph{block} und \emph{interval} Graphen, als auch für \emph{Graphen mit beschränkter mim-width} und \emph{Graphen mit beschränkter clique-width}~\cite{Kloks2021, Galby2020,Courcelle1990,Henning2022,Henning2019}.

Nach einer umfassenden Analyse zum Stand des Problems, beginnen wir eine systematische Analyse aus Sicht der \textit{parametrisierten Komplexität} und zeigen, dass \sdomDE bei einer parameterisierung durch die Größer der Lösungsmenge $\WTWOhs$-hart für \textit{bipartite} und \textit{split} Graphen ist.
Basierend auf vorangegangener Arbeiten~\cite{Alber2004,Garnero2018} war es uns möglich Reduktionsregeln anzugeben, die einen linearen Problemkern der Größe $\kernelsize \cdot k$ für plättbare Graphen erzeugt.
Dies vervollständigt existierende lineare Problemkerne ähnlicher Problem wie \cdom, \rbdom, \efdom, \eddom, \idom oder \dirdom \cite{Diekert2005,Garnero2018,Guo2007,Garnero2017,Luo2013,Alber2006}.

\textbf{Schlagworte}: Stabile Menge; Halbtotale Stabile Menge; Parameterisierte Komplexität; Plättbare Graphen; Linearer Problemkern; Problemreduktion; Graph Theorie
}
\end{abstract}