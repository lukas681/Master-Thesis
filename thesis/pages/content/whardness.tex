
\section{Fixed-Parameter Intractability}

We will start by giving some intractability results and show that \sdom parameterized by the natural parameterization is \WTWOhs-hard on \textit{bipartite graphs} and \textit{triangle-free graphs} by fpt-reducing from \dom on bipartite graphs
\dom on bipartite graphs is known to be \WTWOhs when parameterized with natural parameterization \cite{Raman2008}.

\begin{figure}[ht]
    \label{fig:bipartiteConstruction}
    \begin{equation*}
        \tikzfig{fig/tikz/wbipartite}
    \end{equation*}
\caption[Construction bipartite]{\textit{Reducing to a bipartite $G'$ from the bipartite graph $K_{m,n}$ by duplicating all vertices and adding exactly two forced witnesses.}}
\end{figure}

\begin{theorem}\label{lemma:bipartite}
    \sdom parameterized by solution size is \WTWOhs-hard when restricted to bipartite graphs.
\end{theorem}

\begin{proof}
    We reduce from \dom and consider only $X, Y \neq \emptyset$.
    Given a bipartite $G = ( X \cup Y, E)$, we construct a bipartite graph $G' = (\{X' \cup Y'\},E')$:
    \begin{enumerate}[topsep=0pt,itemsep=0ex,partopsep=1ex,parsep=1ex]
        \item For each $x_i \in X$, we add a new vertex $a_i \in A$  and an edge $\{x_i, a_i\}$ in between. 
        \item For each $y_j \in Y$, we add a new vertex $b_j \in B$ and an edge $\{y_j, b_j\}$ in between.
        \item We add two $P_2$'s with the vertices $u_j,d_j$ and connect their ends to all $a_i$ (resp. $b_i$).
    \end{enumerate}

    $G'$ is bipartite because $A$ and $B$ form an independent set on $G'$ that can be cross-wise attached to $X$ and $Y$. 
    $X' = X \cup \{u_2,d_1\} \cup B$ and $Y' = X \cup \{u_1,d_2\} \cup A$ form the partitions of the new bipartite $G'$.
    It is left to show that $G$ has a ds $D$ of size $k$ if and only if $G'$ has an sds $D'$ of size $k' = k + 2$.
 
    First, assume a ds $D$ in $G$ of size $k$. 
    We know that $D' = D\cup \{d_1,d_2\}$ is an sds in $G'$ of size $k' = k + 2$, because $d_1$ dominates $u_1$ and all $a_i \in A$; $d_2$ dominates $u_2$ and all $b_i \in B$. 
    The same vertices dominate the rest as they were in $G$, but now they all have either $d_1$ or $d_2$ as a witness.
    Therefore,  $\forall v \in (D \cap X) \cup (D \cap Y): (d(v, d_1) = 2 \vee d(v, d_2)=2)$.

    Contrary, assume an sds $D'$ in $G'$ with size $k'$. 
    Wlog, assume that $d_1, d_2 \in D$ and $u_1,u_2 \notin D$, because choosing $d_i$ is preferred to $u_i$.
    By technical assumption $X, Y \neq \emptyset$, $u_i$ can not be a witness for $d_i$.
    If $a_i,b_i \in D'$ we replace it with $x_i$ and $y_i$ preserving the size $D$.
    A $a_i \in A$ can only be used to dominate their neighboring $x_i$ ($b_i \in B$ for $y_i$), because $\abs{N(a_i)} = 2$ and $d_1,d_2\in D'$.

    As $d_1$ and $d_2$ suffice to provide a witness for every vertex in the graph, this operation is sound.
    Hence, $D = D' \setminus \{ d_1,d_2\}$ gives a ds in $G$ of size $ k = k' + 2$.

    As $G'$ can be constructed in linear time and $k$ does not depend on the input size, this reduction is an fpt reduction.  
    Because \dom is \WTWOhs-hard on bipartite graphs~\cite[Th. 1]{Raman2008}, we conclude that \sdom is \WTWOhs-hard when parameterized by solution size as well.
\end{proof}

We will prove intractability for \textit{split} graphs. 
We use the fact that every dominating set in a split graph can directly be mapped to a corresponding semitotal dominating set.

\begin{theorem}\label{lemma:splitgraph}
    \sdom is \WONEhs-hard when restricted to \textit{split} and \textit{chordal} graphs.
\end{theorem}

\begin{proof}

    We reduce from \dom on \textit{split} graphs by showing that any ds in a split graph can be mapped to a sds on the same graph. 
    Given a split graph $G = \{V = (K \cup I), E\}$ with $\abs{V} \geq 2$ and a ds $D$ of size $k$, we can immediately obtain an sds $D'$ by flipping a few vertices:
    If $I \cap D \neq \emptyset$ we replace them respectively with one arbitrary neighbor in $K$.
    All vertices in $I$ are still being dominated and $D \cap K \neq \emptyset$ is sufficient to preserve domination $K$. 
    Note that it is necessary to catch the case, where $D \subseteq I$ and witnesses are missing, which will only be guaranteed after the flip operation. (see \cref{fig:splitgraph}).
    We now assume $D \subseteq K$ and set $D = D'$.
    If $|D'| > 1$, we immediately obtain an sds as $D' \subseteq K$ and $k_1,k_2\in K$ witness each other.
    If $D' = \{d\}$, we add any neighbor of $d$ to $D'$.
    
    For the sake of completeness, if $|V| \leq 1$, we instantly reject as there is no witness available.
    In all cases $k' \leq k + 1$ and because \dom is \WTWOhs-hard on \textit{split} graphs~\cite{Raman2008} the claim follows.
\end{proof}

\begin{figure}[ht]
    \begin{equation*}
        \tikzfig{fig/tikz/split}
    \end{equation*}
\caption[Constructing split graph]{\textit{Obtaining an sds $D'$ form a given ds $D$ in a split graph by flipping all vertices $d_i$ to their corresponding neighbor in the clique $K$.
The clique $K$ is highlighted in {\setulcolor{TUMBlue}\ul{blue}}
Note that in $D$, no witnesses are available as $d(d_1,d_5) = 3$.
After the flip operation, this is fixed.}} \label{fig:splitgraph}
\end{figure}

 
 %Although the previous result implies $w[2]$-hardness for chordal graphs, we found another reduction from \dom on chordal graphs.
 %We will introduce the notion of elimination ordering.
 
 %\begin{definition}[{\cite[Theorem 1]{Rose1960}}]
 %    In a graph \G with n vertices, a vertex is called \textbf{simplicial} if and only if the subgraph of $G$ induced by the vertex set $\{v\} \cup N(v)$ is a complete graph.
 
 %    $G$  is said to have a \textbf{perfect eliminiation ordering} if and only if there is an ordering $(v_1, ... v_n)$ of the vertices, such that each $v_i$ is simplicial in the subgraph induced by the vertices $v_1,...,v_i\{\}$
 %\end{definition}
 
 %The following lemma shows that 
 
 %\begin{lemma}[{\cite[Theorem 1]{Rose1960}}]
 %        A graph \G is chordal if and only if $G$ has a perfect elimination ordering.
 %\end{lemma}

 %\begin{definition}[s-chordal]
 %A graph is s-chordal if it contains at least one chordless cycle of length $s$ and does not contain a chordless cycle of length at least $s+1$
 %\end{definition}
 
 
 %\begin{theorem}
 %    \sdom restricted to $s$-chordal graphs is $\omega[2]$-hard.
 %\end{theorem}
 
 %\begin{figure}[!ht]
 %    \label{fig:chordalReduction}
 %    \begin{equation*}
 %        \tikzfig{fig/tikz/chordal-reduction}
 %    \end{equation*}
 %\caption[Constructing a $s+1$ chordal $G'$ from $s$-chordal $G$]{\textit{Constructing a $s+1$-chordal $G'$ from the $s$-chordal graph $P_4$ by adding a $K_5$, connecting its vertices pairwise to $G$. Adding the (blue) auxiliary vertices are necessary to preserve chordality.}}
 %\end{figure}
 %\begin{proof}
 %    Given an $(s-1)$-chordal graph \G, we construct an $s$-chordal graph $G'$ as described below:
 %    \begin{enumerate}
 %        \item Add a complete graph $K_{n+1}$ and connect each vertex $v \in V$ with exactly one vertex of $K$.
 %        We denote the extant vertex as $u$.
 %        \item Add one additional vertex $t$ and connect it to $u$.
 %        \item For all vertices $v_i \in V$ in $G$, add a new edge $\{n, k_i\}$ for all neighbors $n \in N(v_i)$.
 %    \end{enumerate}
 %    An example reduction on the graph $P_4$ is shown in \cref{fig:chordalReduction}.

 %    \begin{corollary}
 %    $G$ is $s-1$-chordal iff $G'$ is $s$-chordal.
 %    \end{corollary}

 %    \begin{subproof}

 %    Assume $G$ chordal. Then exists a total elimination order $o = (v_1, ..., v_n)$ in G where removing $v_j$ sequentially returns cliques in $N(v_i)$.
 %    Define $o' = (v_1, ..., v_n, k_1, ..., k_n, u, t)$. Applying \cref{cliqueNeighbor} states that $(v_1, ... v_n)$ always gives cliques in G and according to corollary \ref*{cliqueNeighbor} also in G'. As the rest is directly part of a clique in G' by definition with an additional vertex of degree 1, o' is a total elimination order for $G'$, hence G' chordal.

 %    Holds as o' is always a total elimination order in G' and removing the complete subgraph $K_{n+1}$ and $u$ gives a total elimination order in G.
 %    \end{subproof}
 
 
 %    \begin{corollary}
 %    $G$ has a ds $D$ of size $k$ iff $G'$ has an sds $D'$ of size $k+1$
 %    \end{corollary}
 %    \begin{subproof}
 %    Assume a ds $D$ of size $k$ in $G$. $D \cup \{u\}$ is an sds in $G'$ of size $k + 1$, because $u$ dominates $t$ and for each $v \in DS: d(v, u) \leq 2$.
 %    Contrary, assume an sds $SD$ in $G'$. To dominate $t$, $u \in SD$ must hold, hence already dominating the complete subgraph $K_{n+1}$. 
 %    If a vertex $k_i \in SD$, we exchange it with $v_i$ not losing the domination property. Taking $D = SD - \{ u \}$ gives our desired ds of size $k$.
 %    \end{subproof}

 %    This is an fpt reduction and the fact that \dom is \WTWOhs on chordal and $s$-chordal graphs \cite{Liu2011} concludes the proof.

 %\end{proof}
 