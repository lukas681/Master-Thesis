% GENERAL ABBREVIATIONS
%\DeclareUnicodeCharacter{0301}{*************************************}

\newcommand{\mc}{\mathcal}
\newcommand{\notimplies}{\;\not\!\!\!\implies}
\newcommand{\zerotext}[2][0pt]{\makebox[#1][l]{\qquad#2}}

% ABBREVIATIONS
\newcommand{\Dvw}   {\ensuremath{\mc{D}}\xspace}
\newcommand{\Dv}    {\ensuremath{\mc{D}_{v}}\xspace}
\newcommand{\Dw}    {\ensuremath{\mc{D}_{w}}\xspace}

\newcommand{\Nthreev}    {\ensuremath{N_3(v)}}

% SETTINGS
\newcommand*{\getUniversity}{Technical University Munich }
\newcommand*{\getFaculty}{Department of Informatics}
\newcommand*{\getTitle}{On the Parameterized Complexity of Semitotal Domination on Graph Classes}
\newcommand*{\getTitleGer}{Über die Parametrisierte Komplexität des Problems der halbtotalen stabilen Menge auf Graphklassen }
\newcommand*{\getAuthor}{Lukas Retschmeier}
\newcommand*{\getDoctype}{Master Thesis}

% TODO Add University as well
\newcommand*{\getSupervisor}{Prof. Debarghya Ghoshdastidar}
\newcommand*{\getAdvisor}{Prof. Paloma Thomé de Lima}
\newcommand*{\getSubmissionDate}{\today}
\newcommand*{\getSubmissionLocation}{\textit{København S}}

% TIKZ
\newcommand{\vertex}{\node[vertex]}

% COLORSKEME
\newcommand{\red}[1]{\textcolor{red}{#1}}
\newcommand{\blue}[1]{\textcolor{blue}{#1}}
\newcommand{\green}[1]{\textcolor{mygreen}{#1}}
\newcommand{\brown}[1]{\textcolor{brown}{#1}}

% THEOREM ENVIRONMENTS
\newtheorem{fact}{\protect\color{TUMSecondaryBlue} Fact}[section] % CHAPTER CHANGE TO SECTION TO GET ONE MORE LEVEL
\newtheorem{rgl}  {\protect\color{TUMSecondaryBlue} Rule} 	
\newtheorem{theorem}{\protect\color{TUMSecondaryBlue} Theorem}

\newtheorem{definition}{\protect\color{TUMSecondaryBlue} Definition}[section]
\newtheorem{graphclass}{\protect\color{TUMSecondaryBlue} Graph Class}

\newtheorem{corollary}{\protect\color{TUMSecondaryBlue} Corollary}[section]
\newtheorem{lemma}{\protect\color{TUMSecondaryBlue} Lemma}[section]
\newtheorem{proposition}{\protect\color{TUMSecondaryBlue} Proposition}[section]

\newcounter{casenum}
\newenvironment{caseof}{\setcounter{casenum}{0}}{\vskip.5\baselineskip}
% \newcommand{\case}[2]{\vskip.5\baselineskip\par\noindent {\bfseries Case \arabic{casenum}:} #1\\#2\addtocounter{casenum}{1}}
\newcommand{\case}[2]{%
  \par\addvspace{.5\baselineskip}%
  \noindent \refstepcounter{casenum}\textbf{Case \thecasenum:}~#1\\*
  #2\ifhmode\unskip\fi
}

\crefname{case}{Case}{Cases}
\crefname{casenum}{\protect\textbf{Case}}{\protect\textbf{Cases}}
\Crefname{casenum}{\protect\textbf{Case}}{\protect\textbf{Cases}}
\creflabelformat{casenum}{#2\textbf{(#1)}#3}

\crefname{chapter}{chapter}{chapters}
\crefname{figure}{figure}{figures}
\crefname{section}{section}{section}

\newenvironment{caseofz}{\setcounter{casenum}{0}}{\vskip.5\baselineskip}
\newcommand{\casez}[2]{\vskip.5\baselineskip\par\noindent {\bfseries Case \arabic{casenum}:} #1\\#2\addtocounter{casenum}{1}}
%\newcommand{\case}[2]{\vskip.5\baselineskip\par\noindent {\bfseries Case \arabic{casenum}:} #1\\#2\addtocounter{casenum}{1}}
%\crefname{case}{Case}{Cases}

\crefname{rgl}{Rule}{Rules}
\crefname{lemma}{Lemma}{Lemmas}
\crefname{fact}{Fact}{Facts}
\crefname{theorem}{Theorem}{Theorems}
\crefname{definition}{Definition}{Definitions}
\crefname{graphclass}{Graph Class}{Graph Classes}
\crefname{corollary}{Corollary}{Corollaries}
\crefname{proposition}{Proposition}{Propositions}
\crefname{figure}{Figure}{Figure}

\Crefname{rgl}{Rule}{Rules}
\Crefname{fact}{Fact}{Facts}
\Crefname{theorem}{Theorem}{Theorems}
\Crefname{definition}{Definition}{Definitions}
\Crefname{graphclass}{Graph Class}{Graph Classes}
\Crefname{corollary}{Corollary}{Corollaries}
\Crefname{figure}{Figure}{Figure}


\newenvironment{subproof}[1][\proofname]{%
  \renewcommand{\qedsymbol}{$\blacksquare$}%
  \begin{proof}[#1]%
}{%
  \end{proof}%
}

% ACRONYMS
\newacronym[description={Problem},%
    name={Semitotal Dominating Set}]{sds}{sds}{Semitotal Dominating Set}

\newcommand{\name}[1]{\textsc{#1}}
\newcommand{\dom}	{\name{Dominating Set}\xspace}
\newcommand{\DOM}	{\name{DOMINATING SET}\xspace}
\newcommand{\pdom}	{\name{Planar Dominating Set}\xspace}
\newcommand{\clique}{\name{Clique}\xspace}
\newcommand{\is}{\name{Independent Set}\xspace}
\newcommand{\wsat}{\name{Weighted Boolean Satisfiabiliy}\xspace}

\newcommand{\tdom}{\name{Total Dominating Set}\xspace}
\newcommand{\tdomin}{\name{Total Domination}\xspace}
\newcommand{\TDOM}{\name{TOTAL DOMINATING SET}\xspace}
\newcommand{\ptdom}{\name{Planar Total Dominating Set}\xspace}

\newcommand{\sdom}{\name{Semitotal Dominating Set}\xspace}
\newcommand{\SDOM}{\name{SEMITOTAL DOMINATING SET}\xspace}
\newcommand{\sdomin}{\name{Semitotal Domination}\xspace}
\newcommand{\msdom}{\name{Minimum Semitotal Dominating Set}\xspace}
\newcommand{\sdomd}{\name{Semitotal Domination Decision}\xspace}
\newcommand{\psdom}{\name{Planar Semitotal Dominating Set}\xspace}

\newcommand{\cdom}{\name{Connected Dominating Set}\xspace}
\newcommand{\pcdom}{\name{Planar Connected Dominating Set}\xspace}

\newcommand{\prbdom}{\name{Planar Red-Blue Dominating Set}\xspace}
\newcommand{\rbdom}{\name{Planar Red-Blue Dominating Set}\xspace}
\newcommand{\dreg}{\textit{$D$-region decomposition}\xspace}

\newcommand{\efdom}{\name{Efficient Dominating Set}\xspace}
\newcommand{\pefdom}{\name{Planar Efficient Dominating Set}\xspace}

\newcommand{\eddom}{\name{Edge Dominating Set}\xspace}
\newcommand{\peddom}{\name{Planar Edge Dominating Set}\xspace}

\newcommand{\dirdom}{\name{Directed Dominating Set}\xspace}
\newcommand{\pdirdom}{\name{Planar Directed Dominating Set}\xspace}

\newcommand{\idom}{\name{Independent Dominating Set}\xspace}
\newcommand{\pidom}{\name{Planar Independent Dominating Set}\xspace}

\newcommand{\rdom}{\name{Roman Dominating Set}\xspace}
\newcommand{\prdom}{\name{Planar Roman Dominating Set}\xspace}

\newcommand{\NPc}{\name{\textbf{NP}-Complete}\xspace}
\newcommand{\NPcs}{\name{\textbf{NP}-c}\xspace}
\newcommand{\NPcn}{\name{\textbf{NP}-Completeness}\xspace}

\newcommand{\SAT}{\name{Boolean Satisfiability}\xspace}
\newcommand{\SATs}{\name{SAT}\xspace}
\newcommand{\NPh}{\name{\textbf{NP}-hard}\xspace}
\newcommand{\NP}{\name{\textbf{NP}}\xspace}
\newcommand{\Pt}{\name{\textbf{P}}\xspace}
\newcommand{\FPT}{\name{\textbf{FPT}}\xspace}
\newcommand{\FPTl}{\name{Fixed-Parameter Tractable}\xspace}
\newcommand{\DTIME}{\name{\textbf{DTIME}}\xspace}
\newcommand{\WONE}{\ensuremath{\mathbf{W[1]}}\xspace}
\newcommand{\WTWO}{\ensuremath{\mathbf{W[2]}}\xspace}
\newcommand{\WTWOhs}{\ensuremath{\mathbf{W[2]}}-h\xspace}
\newcommand{\WHIERARCHY}{\ensuremath{\mathbf{W}}-hierarchy\xspace}
\newcommand{\Wt}{\ensuremath{\mathbf{W[t]}}\xspace}

\newcommand{\gi}{\name{Graph Isomorphism}\xspace}
\newcommand{\G}		{\ensuremath{G=(V,E)}\xspace}
\newcommand{\GB}	{\ensuremath{G'=(V',E')}\xspace}
\newcommand{\R}		{\ensuremath{R(v,w)}\xspace}
\newcommand{\N}[1]	{\ensuremath{N_{#1}(v,w)}\xspace}
\newcommand{\DR}	{\ensuremath{\mc{R}}\xspace}
\newcommand{\vw}	{\ensuremath{\{ v,w\}}\xspace}
\newcommand{\sr}	{simple region}

\newcommand{\bg}	{bipartite graphs}
\newcommand{\cg}	{chordal graphs}
\newcommand{\cbg}	{chordal bipartite graphs}
\newcommand{\rpg}	{\ensuremath{r}-partite graphs}
\newcommand{\sg}	{split graphs}

\newcommand{\dGR} {\ensuremath{d_{G_{\mathfrak{R}}}}}
\newcommand{\GR} {\ensuremath{{G_{\mathfrak{R}}}}}
\newcommand{\VR} {\ensuremath{{V(\mathfrak{R})} } }

\newcommand{\wtwo}	{\ensuremath{w[2]}}
\newcommand{\wone}	{\ensuremath{w[1]}}

\newcommand{\noni}{\ensuremath{\abs{N_1(v,w) \cap V(R)}}}
\newcommand{\ntwi}{\ensuremath{\abs{N_2(v,w) \cap V(R)}}}
\newcommand{\nthi}{\ensuremath{\abs{N_3(v,w) \cap V(R)}}}

% CONSTANTS
\newcommand\kernelsize{359}

\newenvironment{abstract}[1]{%
\begin{center}\normalfont\textbf{Abstract}\end{center}
\begin{quotation} #1 \end{quotation}
}{%
\vspace{1cm}
}


\ifstrempty{#1}%
% if condition (without title)


%%%%%%%%%%%%%%%%%%%%%%%%%%%%%%
%Theorem
\newcounter{theo}[section] \setcounter{theo}{0}
\renewcommand{\thetheo}{\arabic{section}.\arabic{theo}}
\newenvironment{theo}[2][]{%
\refstepcounter{theo}%
\ifstrempty{#1}%
{\mdfsetup{%
frametitle={%
\tikz[baseline=(current bounding box.east),outer sep=0pt]
\node[anchor=east,rectangle,fill=blue!20]
{\strut Theorem~\thetheo};}}
}%
{\mdfsetup{%
frametitle={%
\tikz[baseline=(current bounding box.east),outer sep=0pt]
\node[anchor=east,rectangle,fill=blue!20]
{\strut Theorem~\thetheo:~#1};}}%
}%
\mdfsetup{innertopmargin=10pt,linecolor=blue!20,%
linewidth=2pt,topline=true,%
frametitleaboveskip=\dimexpr-\ht\strutbox\relax
}
\begin{mdframed}[]\relax%
\label{#2}}{\end{mdframed}}

%%%%%%%%%%%%%%%%%%%%%%%%%%%%%%
%Lemma
\newcounter{lem}[section] \setcounter{lem}{0}
\renewcommand{\thelem}{\arabic{section}.\arabic{lem}}
\newenvironment{lem}[2][]{%
\refstepcounter{lem}%
\ifstrempty{#1}%
{\mdfsetup{%
frametitle={%
\tikz[baseline=(current bounding box.east),outer sep=0pt]
\node[anchor=east,rectangle,fill=green!20]
{\strut Lemma~\thelem};}}
}%
{\mdfsetup{%
frametitle={%
\tikz[baseline=(current bounding box.east),outer sep=0pt]
\node[anchor=east,rectangle,fill=green!20]
{\strut Lemma~\thelem:~#1};}}%
}%
\mdfsetup{innertopmargin=10pt,linecolor=green!20,%
linewidth=2pt,topline=true,%
frametitleaboveskip=\dimexpr-\ht\strutbox\relax
}
\begin{mdframed}[]\relax%
\label{#2}}{\end{mdframed}}



%%%%%%%%%%%%%%%%%%%%%%%%%%%%%%
%Proof
\newcounter{prf}[section]\setcounter{prf}{0}
\renewcommand{\theprf}{\arabic{section}.\arabic{prf}}
\newenvironment{prf}[2][]{%
\refstepcounter{prf}%
\ifstrempty{#1}%
{\mdfsetup{%
frametitle={%
\tikz[baseline=(current bounding box.east),outer sep=0pt]
\node[anchor=east,rectangle,fill=red!20]
{\strut Proof~\theprf};}}
}%
{\mdfsetup{%
frametitle={%
\tikz[baseline=(current bounding box.east),outer sep=0pt]
\node[anchor=east,rectangle,fill=red!20]
{\strut Proof~\theprf:~#1};}}%
}%
\mdfsetup{innertopmargin=10pt,linecolor=red!20,%
linewidth=2pt,topline=true,%
frametitleaboveskip=\dimexpr-\ht\strutbox\relax
}
\begin{mdframed}[]\relax%
\label{#2}}{\qed\end{mdframed}}

%%%%%%%%%%%%%%%%%%%%%%%%%%%%%%
% Problem
\newcounter{prb}[section]\setcounter{prb}{0}
\renewcommand{\theprf}{\arabic{prb}}
\newenvironment{prb}[2][]{%
\refstepcounter{prb}%
\ifstrempty{#1}%
{\mdfsetup{%
frametitle={%
\tikz[baseline=(current bounding box.east),outer sep=0pt]
\node[anchor=east,rectangle,fill=red!20]
{\strut PROBLEM~\theprf};}}
}%
{\mdfsetup{%
frametitle={%
\tikz[baseline=(current bounding box.east),outer sep=0pt]
\node[anchor=east,rectangle,fill=red!20]
{\strut ~#1};}}%
}%
\mdfsetup{innertopmargin=10pt,linecolor=red!20,%
linewidth=2pt,topline=true,%
frametitleaboveskip=\dimexpr-\ht\strutbox\relax
}
\begin{mdframed}[]\relax%
\label{#2}}{\end{mdframed}}