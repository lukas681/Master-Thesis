\PassOptionsToPackage{table,svgnames,dvipsnames}{xcolor}

% Define TUM corporate design colors
% Taken from http://portal.mytum.de/corporatedesign/index_print/vorlagen/index_farben
\definecolor{TUMBlue}{HTML}{0065BD}
\definecolor{TUMSecondaryBlue}{HTML}{005293}
% \definecolor{TUMLightBlue}{HTML}{005293}
\definecolor{TUMSecondaryBlue2}{HTML}{003359}
\definecolor{TUMBlack}{HTML}{000000}
\definecolor{TUMWhite}{HTML}{FFFFFF}
\definecolor{TUMDarkGray}{HTML}{333333}
\definecolor{TUMGray}{HTML}{808080}
\definecolor{TUMLightGray}{HTML}{CCCCC6}
\definecolor{TUMAccentGray}{HTML}{DAD7CB}
\definecolor{TUMAccentOrange}{HTML}{E37222}
\definecolor{TUMAccentGreen}{HTML}{A2AD00}
\definecolor{TUMAccentLightBlue}{HTML}{98C6EA}
\definecolor{TUMAccentBlue}{HTML}{64A0C8}
\definecolor{VeryLightBlue}{HTML}{E8F8FF}


% Used for graph classes
\definecolor{darkred}{HTML}{E80000}
\definecolor{darkgreen}{HTML}{006400} 
\definecolor{yellowgreen}{HTML}{9ACD32}
% \definecolor{gray}{HTML}{E8F8FF}


\usepackage[utf8]{inputenc}
\usepackage[T1]{fontenc}
\usepackage[sc]{mathpazo}
\usepackage[ngerman,american]{babel}

\usepackage[autostyle]{csquotes}

\usepackage[%
  backend=biber,
  url=false,
  sortcites=true,
  style=numeric,
  maxnames=4,
  minnames=3,
  maxbibnames=99,
  giveninits,
  uniquename=init]{biblatex}
\usepackage{hyperref} % hidelinks removes colored boxes around references and links
\hypersetup{
colorlinks=true,
linkcolor=darkgray,
citecolor=TUMBlue,
%TODO pdftitle={Your PDF title},
pdfsubject={Parameterized Complexity},
pdfauthor={Lukas Retschmeier},
pdfkeywords={ParameterizedComplexity, Algorithms, Kernelization}
}

\usepackage{graphicx}
\usepackage{scrhack} % necessary for listings package
\usepackage{listings}
\usepackage{lstautogobble}
\usepackage{tikz}
\usepackage{pgfplots}
\usepackage{pgfplotstable}
\usepackage{booktabs}
\usepackage[final]{microtype}
\usepackage{caption}

\bibliography{bibliography}

\setkomafont{disposition}{\normalfont\bfseries} % use serif font for headings
\linespread{1.05} % adjust line spread for mathpazo font

% Add table of contents to PDF bookmarks
\BeforeTOCHead[toc]{{\cleardoublepage\pdfbookmark[0]{\contentsname}{toc}}}

% Settings for pgfplots
\pgfplotsset{compat=newest}
\pgfplotsset{
  % For available color names, see http://www.latextemplates.com/svgnames-colors
  cycle list={TUMBlue\\TUMAccentOrange\\TUMAccentGreen\\TUMSecondaryBlue2\\TUMDarkGray\\},
}

% Settings for lstlistings
\lstset{%
  basicstyle=\ttfamily,
  columns=fullflexible,
  autogobble,
  keywordstyle=\bfseries\color{TUMBlue},
  stringstyle=\color{TUMAccentGreen}
}

% Epigraphs
\setlength\epigraphwidth{.6\textwidth}
\setlength\epigraphrule{0pt}

% TIKZIT / TIKZ STYLING
\input{styling/tikz.tikzstyles}

\DeclareCaptionType{equ}[][]
\DeclareCaptionLabelFormat{blank}{}

% Reducing Spacing in new chapter
\usepackage{etoolbox}
\makeatletter
\patchcmd{\@makechapterhead}{\vspace*{50\p@}}{\vspace*{-20\p@}}{}{}
\patchcmd{\@makeschapterhead}{\vspace*{50\p@}}{\vspace*{-20\p@}}{}{}
\patchcmd{\DOTI}{\vskip 80\p@}{\vskip 0\p@}{}{}
\patchcmd{\DOTIS}{\vskip 40\p@}{\vskip 0\p@}{}{}
\makeatother
% -------------------------------------------------------------------------------------------

% REQURES FANCYHDR, BUT I DONT LIKE IT ANYMORE
% Definition der Kopf- und Fußzeilen
%\lhead{}								% Kopf links
%\chead{}								% Kopf mitte
%\rhead{\sffamily{\leftmark}}				% Kopf rechts
%\lfoot{}								% Fuß links
%\cfoot{\sffamily{\thepage}}				% Fuß mitte
%\rfoot{\sffamily{\autor}}				% Fuß rechts
%\renewcommand{\headrulewidth}{0.4pt}	% Liniendicke Kopf
%\renewcommand{\footrulewidth}{0.4pt}	% Liniendicke Fuß
%\fancyhead{}
%\lhead{}
%\chead{}
%\rhead{\textbf{\rightmark}}
%\lfoot{Lukas Retschmeier}
%\cfoot{}
%\rfoot{\thepage/\pageref{LastPage}}
%\renewcommand{\headrulewidth}{0.4pt}
%\renewcommand{\footrulewidth}{0.4pt}

% \let\thempfootnote\thefootnote
