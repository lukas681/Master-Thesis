\documentclass[
	%sans,			% use sans-serif font
	%serif,			% use serif-font
	%mathsans,		% set mathtext to sans-serif
	%mathserif,		% set mathtext to serif
	%10pt,
	10pt,
	%12pt,
	t		% add text at the top border of slide
	%slidescentered,% center text on slide
	%draft,			% compile as draft version
	%handout,		% create handout file
	%notes,			% include nodes in slides
	%compress		% compress navigation bar
]{beamer}


\usetheme{lmtslides}
\usepackage{eso-pic}
\usepackage{graphicx}
%\usepackage[pdftex]{color}
\usepackage{times}
\usepackage[latin1]{inputenc}
%\usepackage[T1]{fontenc}
\usepackage[amssymb]{SIunits}
\usepackage{amsmath,amssymb}
\usepackage{eurosym}
\usepackage{booktabs}
\usepackage{colortbl}
\usepackage{url}

\graphicspath{{figures/}}

% SET LANGUAGE HERE! (Babel is already included and setup by this call.)
\setlang{de}		% <- GERMAN
%\setlang{en}		% <- ENGLISH

% MODIFY THESE ACCORDINGLY! ---
\title{Hier steht der komplette Titel der Arbeit}
\subtitle{Diplomarbeit Abschlussvortrag}
\type{Mf} % (M/B/D/S)(f/m): (Master/Bachelor/Diplom/Studienarbeit)(final/midterm)
\author{Vorname Nachname}
\email{email@mytum.de}
\advisor{Vorname Nachname}
\emailAdvisor{email@tum.de}
\date{\today}
%------------------------------


%%%%%%%%%%%%%%%%%%%%%%%%%%
\begin{document}

\AddToShipoutPicture{\TitlePicture}
\maketitle
\ClearShipoutPicture
\AddToShipoutPicture{\BackgroundPicture}

\section{�berblick der Arbeit}
\begin{frame}
\frametitle{�berblick der Arbeit}
\begin{itemize}
\item Zielsetzung:
\begin{itemize}
\item \dots
\end{itemize}
\item Arbeitsschritte/Aufgaben:
\begin{itemize}
\item Modellierung von ...
\item Simulation mit ...
\item ...
\end{itemize}
\end{itemize}

\end{frame}
\section{Kurz�berblick zur Arbeit}
\subsection{Zusammenfassung der Ergebnisse}
\begin{frame}
\frametitle{Erzielte Ergebnisse}
\begin{itemize}
\item ...
\item ...
\item ...
\end{itemize}

\end{frame}

\subsection{Bisherige Arbeiten, Referenzen und Aufgaben}
\begin{frame}
\frametitle{Bisherige Arbeiten, Referenzen und Aufgaben}
\begin{itemize}
\item Die Arbeit basiert auf:
\begin{itemize}
\item ...
\end{itemize}
\item Was war bereits verf�gbar:
\begin{itemize}
\item ...
\end{itemize}
\item Was ist neu bzw. war neu zu erstellen:
\begin{itemize}
\item  ...
\end{itemize}
\end{itemize}

\end{frame}


\section{Vortragsfolien}
\begin{frame}
\frametitle{Vortragsfolien}
\end{frame}

\end{document}
