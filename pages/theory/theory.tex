\chapter{Preliminaries}
We start by recapping some basic notation in Graph Theory and Parametrized Complexity. 

%% TODO State Draft
Continuing an intensive study of parametrized complexity of that problem. 

\section{Graph Theory}
We quickly state the following definitions given by {\cite[p.~xxx]{diestel10}}.

%% TODO Path, Subgraph, Induced Subgraph, Copnnected 
\begin{definition}[Graph {\cite[p. 3]{diestel10}}]
    A graph is a pair $G = (V, E)$ of two sets where $V$ denotes the vertices and $E \subseteq V \times V$ the edges of the graph.  A vertex $v \in V$ is incident with an edge $e \in E$ if $v \in e$. Two vertices $x, y$ are adjacent, or neighbours, if $\{x,y \} \in E$.
    % Size of the Graph?
\end{definition}

%TODO Quote
\begin{definition}[Vertex Degrees]
    The \textit{degree} $d_G(v)$ (If G is clear, also $d(v)$) of a vertex $v$ is the number of neighbors of v. We call a vertex of degree 0 as isolated and one of degree 1 as a pendant.
\end{definition}

\begin{definition}[isomorphic Graphs, {\cite*[p. 3]{diestel10}}]
Let \G and $G' = (V', E')$ be two graph. We call $G$ and $G'$ \underline{isomorphic}, if there exists a bijection $\phi: V \rightarrow V'$ with $\{x, y\} \in E \Leftrightarrow \phi(x)\phi(y) \in E'$ for all  $x,y \in V$. Such a map $\phi$ is called \underline{isomorphism}.
\end{definition}

\begin{definition}[Special Graph Notations {\cite[p.~27]{diestel10}}]
    A simple Graph
    
    A directed Graph is a graph
    
    A Multi Graph
    
    A Planar Graph
\end{definition}

% TODO Quote e.g. The open neighborhood number of a graph
\begin{definition}[Closed and Open Neighborhoods {\cite{Balakrishnan2012}}]
    Let \G be a (non-empty) graph. 
    The set of all neighbors of $v$ is the \underline{open neighborhood} of $v$ and denoted by $N(v)$; the set $N[v] = N(v) \Cup \{v\}$ is the \underline{closed neighborhood} f $v$ in $G$. When G needs to be made explicit, those open and closed neighborhoods are denoted by $N_G(v)$ and $N_G[v]$. 
\end{definition}

\begin{definition}[Induced Subgraph]
    asd
\end{definition}

\begin{definition}[Isomorphic Graph]
    asd
\end{definition}

\subsection*{Special Graph Classes}
We call the class of graphs without any special restrictions ``General Graphs''.

\begin{definition}[r-partite Graphs]
    Let $r \geq 2$ be an integer. A Graph $G = (V,E)$ is called ``r-partite'' if V admits a partition into r classes such that every edge has its ends in different classes: Vertices in the same partition class must not be adjacent. 
    
    For the case $r = 2$ we say that the G is ``bipartite'' 
    %      %        TODO mage of a bipartite Graph
    
\end{definition}

\begin{definition}[Chordal Graphs]
    
\end{definition}

\begin{definition}[Split Graphs]
    
\end{definition}

\begin{definition}[Planar Graphs {\cite[Chapter 4]{diestel10}}]

A \textit{plane graph} is a pair $(V,E)$ of finite sets with the following properties:

\begin{itemize}
    \item $V \subseteq \mathbb{R}^2$ (Vertices)
    \item Every edge is an arc between two vertices 
    \item different edges have different sets of endpoints
    \item The interior of an edge contains no vertex and no point of any other edge
\end{itemize}

An embedding in the plane, or \textit{planar embedding}, of an (abstract) graph $G$ is an isomorphism between $G$ and a plane graph $H$. A \textit{plane graph} can be seen as a concrete \textbf{embedding} of the planar graph into the ``plane'' $\mathbb{R}^2$.

\end{definition}

% Independent Set

\section{Parametrized Complexity}
We are now giving a short introduction into the world of parametrized complexity. 
* General Introduction
\paragraph{Ways to cope with NP-hard problem.}

\subsection{Fixed Parameter Tractability}
\paragraph{Fixed Parameter Intractability: The \hmath $W$ Hierarchy}

\subsection{Kernelization}
\subsubsection{Formal Definitions}
\begin{definition}[TODO CITE]
A \textit{Kernelization Algorithm} or \textit{kernel} is an algorithm $\mathfrak{A}$ for a parametrized Problem Q, that given an instance $(I,k)$ of Q works in polynomial timeand returns an equivalent instance $(I', k')$ of Q. Moreover, we require that $size_{\mathfrak{A}}(k) \leq g(k)$ for some computable function $g:\mathcal{N} \rightarrow \mathcal{N}$
\end{definition}


\chapter{On Parametrized Semitotal Domination}

\vspace*{-50pt}

\begin{figure}[ht]
        \includegraphics[width=0.35\textwidth, right]{img/placeholder.png}
        \captionsetup{textformat=empty,labelformat=blank}
        \caption{Generated with Dall-E. \url{https://labs.openai.com/}. ``A duck dominating sitting on a searose''}
\end{figure}

\epigraph{\itshape Todo select another quote}{Lewis Caroll, \textit{XXXX}}


\section{Semitotal Domination}

\textbf{Problem Definition}

\begin{prb}[DOMINATING SET]{prb:ds}

~\\
    
    \begin{tabularx}{0.9\textwidth}{>{\hsize=0.30\hsize}X>{\hsize=0.7\hsize}X}
        \textbf{Input:} & Graph \G and an integer $k$\\
        \textbf{Question:} & Is there a set $X \subseteq V$ of size $k$ s.t. $N[X] = V(G)$? \\
    \end{tabularx}
        
\end{prb}

\begin{prb}[DOMINATING SET]{prb:tds}
    
    \begin{tabularx}{0.8\textwidth}{>{\hsize=0.35\hsize}X>{\hsize=0.65\hsize}X}
        \textbf{Input:} & Graph \G and an integer $k$\\
        \textbf{Question:} & TODO \\
    \end{tabularx}
        
\end{prb}

\begin{prb}[SEMITOTAL DOMINATING SET]{prb:sds}
    
    \begin{tabularx}{0.8\textwidth}{>{\hsize=0.35\hsize}X>{\hsize=0.65\hsize}X}
        \textbf{Input:} & Graph \G and an integer $k$\\
        \textbf{Question:} & TODO \\
    \end{tabularx}
        
\end{prb}




Dominating Set



Semitotal Dominating Set

Total Dominating Set

we denote y as the dominating number. Clearly $y_t < y_s < y_d$.

\subsection{Preliminaries}

* Witness
* domination 

Let $D$ be a dominating set of G and $w \in V(G) \setminus D$. For any neighbor $v \in D \cap N(w)$, we say that $d_1$ \textit{dominates} $w$ For two dominating vertices $d_1, d_2in D$. If 

\sdom

%\subsection{Some useful notation}


Definition, dominating number

\section{Complexity Status of \sdom}

\section{\hmath $w[i]$-Intractibility}

Now some w[i] hard classes. 

\subsection{Warm-Up: \hmath $W[2]$-hard on General Graphs}

% TODO can we conclude anything for AT Free Graphs?
%% TODO Extend to r-partite

As any \bg with bipartition can be split further into \rpg this results also implies the \wone-hardness of \rpg

