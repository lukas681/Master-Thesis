\chapter{Preliminaries}
We start by recapping some basic notation in Graph Theory and Parametrized Complexity. 

%% TODO State Draft
Continuing an intensive study of parametrized complexity of that problem. 

\section{Graph Theory}
We quickly state the following definitions given by {\cite[p.~xxx]{diestel10}}.

%% TODO Path, Subgraph, Induced Subgraph, Copnnected 
\begin{definition}[Graph {\cite[p. 3]{diestel10}}]
    A graph is a pair $G = (V, E)$ of two sets where $V$ denotes the vertices and $E \subseteq V \times V$ the edges of the graph.  A vertex $v \in V$ is incident with an edge $e \in E$ if $v \in e$. Two vertices $x, y$ are adjacent, or neighbours, if $\{x,y \} \in E$.
    % Size of the Graph?
\end{definition}

%TODO Quote
\begin{definition}[Vertex Degrees]
    The \textit{degree} $d_G(v)$ (If G is clear, also $d(v)$) of a vertex $v$ is the number of neighbors of v. We call a vertex of degree 0 as isoliated and one of degree 1 as a pendant.
\end{definition}

\begin{definition}[isomorphic Graphs, {\cite*[p. 3]{diestel10}}]
Let \G and $G' = (V', E')$ be two graph. We call $G$ and $G'$ \underline{isomorphic}, if there exists a bijection $\phi: V \rightarrow V'$ with $\{x, y\} \in E \Leftrightarrow \phi(x)\phi(y) \in E'$ for all  $x,y \in V$. Such a map $\phi$ is called \underline{isomorphism}.
\end{definition}

\begin{definition}[Special Graph Notations {\cite[p.~27]{diestel10}}]
    A simple Graph
    
    A directed Graph is a graph
    
    A Multi Graph
    
    A Planar Graph
\end{definition}

% TODO Quote e.g. The open neighborhood number of a graph
\begin{definition}[Closed and Open Neighborhoods {\cite{Balakrishnan2012}}]
    Let \G be a (non-empty) graph. 
    The set of all neighbors of $v$ is the \underline{open neighborhood} of $v$ and denoted by $N(v)$; the set $N[v] = N(v) \Cup \{v\}$ is the \underline{closed neighborhood} f $v$ in $G$. When G needs to be made explicit, those open and closed neighborhoods are denoted by $N_G(v)$ and $N_G[v]$. 
\end{definition}

\begin{definition}[Induced Subgraph]
    asd
\end{definition}

\begin{definition}[Isomorphic Graph]
    asd
\end{definition}

\subsection*{Special Graph Classes}
We call the class of graphs without any special restrictions ``General Graphs''.

\begin{definition}[r-partite Graphs]
    Let $r \geq 2$ be an integer. A Graph $G = (V,E)$ is called ``r-partite'' if V admits a parititon into r classes such that every edge has its ends in different classes: Vertices in the same partition class must not be adjacent. 
    
    For the case $r = 2$ we say that the G is ``bipartite'' 
    %      %        TODO mage of a bipartite Graph
    
\end{definition}

\begin{definition}[Chordal Graphs]
    
\end{definition}

\begin{definition}[Split Graphs]
    
\end{definition}

% Independent Set

\section{Parametrized Complexity}
We are now giving a short introduction into the world of parametrized complexity. 
* General Introduction
\paragraph{Ways to cope with NP-hard problem.}

\subsection{Fixed Parameter Tractability}
\paragraph{Fixed Parameter Intractability: The \hmath $W$ Hierarchy}

\subsection{Kernelization}
\subsubsection{Formal Definitions}
\begin{definition}[TODO CITE]
A \textit{Kernelization Algorithm} or \textit{kernel} is an algorithm $\mathfrak{A}$ for a parametrized Problem Q, that given an instance $(I,k)$ of Q works in polynomial timeand returns an equivalent instance $(I', k')$ of Q. Moreover, we require that $size_{\mathfrak{A}}(k) \leq g(k)$ for some computable function $g:\mathcal{N} \rightarrow \mathcal{N}$
\end{definition}


\chapter{On Parametrized Semitotal Domination}
\section{Semitotal Domination}

\sdom

%TODO improve definition
For two dominating vertices $d_1, d_2$ we say that they are wittnesses for each other if $d(d_1, d_2) \leq 2$
%\subsection{Some useful notation}


Definition, dominating number

\subsection*{Complexity Status of \sdom}

\section{\hmath $w[i]$-Intractibility}

Now some  w[i] hard classes. 

\subsection{Warm-Up: \hmath $W[2]$-hard on General Graphs}

% TODO can we conclude anything for AT Free Graphs?
%% TODO Extend to r-partite

As any \bg with bipartition can be split further into \rpg this results also implies the \wone-hardness of \rpg

