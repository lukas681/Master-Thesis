\chapter{Introduction}\label{ch:introduction}

Parametrized Complexity emerging branch. Books about that

Semitotal domination introduced by 
% TODO: Idea for a nice introduction 

Idea:  Lake with stones, and family of ducks of fixed size wants to occupy the lake so that no other clan tries to take it over.
Rules: 
* A duck can quack freeing up neighboring stones.
* Ducks don't like to be alone and want to quack together. So for every duck their must be another duck that is not further than two stones away.
Q: Can our ducklings occupy the whole lake while not feeling lonely?


\section{Content of the thesis}

In this thesis we continue the systematic analysis of the \sdom problem by focusing on the parametrized complexity of the problem. 

Although the problem already had a lot of attention regarding classical complexity (CITE), only a few results are currently known for the parametrized variant. 

As far as we have seen, even the w-hardness of the general case has not been explicitly been proofen in the literature. 

In this thesis, we continue the journey towards a systematic analysis by stating some hardness results for specific graph classes for the problem.

\paragraph{Our contributions}
% TODO Better: 

Our main contributations consist of first showing the $w[2]$-hardness of \sdom for XXXX graphs.

\noindent As the \dom problem and the \tdom problem both admit a linear kernel for planar graphs, it is interesting to analyse wether this results also holds for the \sdom problem which lays in between these two. 
%TODO by relxing the witness of these two provlemsproblems.

Having these kernels also for other variants like \eddom, \efdom, \cdom, \rbdom lent us a great confidence that the result will also work for \sdom on planar graphs.

%% TODO Find more  .

Following the approach from ... which alraedy relies on the technique given in, we give some simple data reduction rules for \sdom on planar graphs leading to a linear kernel. More precisely, we are going to proof the following central theorem of this thesis:

With some  modifications we were able to transfer the approach given by Garnero and Sau in \cite{Garnero2018} to the \sdom problem.

\begin{restatable}[]{theorem}{centraltheo}\label{thm:central}
    The \sdom problem parametrized by solution size admits a linear kernel on planar graphs. There exists a polynomial-time algorithms that given a planar graph $(G, k)$, either correctly reports that $(G, k)$ is a NO-instance or returns an equivalent instance $(G', k)$ such that $\abs{V(G')} \leq \kernelsize \cdot k$.
\end{restatable}

\dom problem and \tdom problem, both already 

