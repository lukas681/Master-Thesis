\chapter{Introduction}

Parametrized Complexity emerging branch. Books about that

Semitotal domination introduced by 
% TODO: Idea for a nice introduction 
\section{Content of the thesis}

In this thesis we continue the systematic analysis of the \sdom problem by focusing on the parametrized complexity of the problem. 

Although the problem already had a lot of attention regarding classical complexity (CITE), only few results are currently known for the parametrized variant. 

As far as we have seen, even the w-hardness of the general case has not been explicitely been proofen in the literature. 

In this thesis we continue the journey towards a systematic analysis by stating some hardness results for specific graph classes for the problem.

\paragraph{Our contributions}
% TODO Better: 

Our main contributations consist of first showing the $w[2]$-hardness of \sdom for XXXX graphs.

\noindent As the \dom problem and the \tdom problem both admit a linear kernel for planar graphs, it is interesting to analyse wether this results also holds for the \sdom problem which lays in between these two. 
%TODO by relxing the witness of these two provlemsproblems.

Having these kernels also for other variants like \eddom, \efdom, \cdom, \rbdom lent us a great confidence that the result will also work for \sdom on planar graphs.

%% TODO Find more.

Following the approach from ... which alraedy relies on the technique given in, we give some simple data reduction rules for \sdom on planar graphs leading to a linear kernel. More precisely, we are going to proof the following central theorem of this thesis:

\begin{restatable}[]{theorem}{centraltheo}\label{thm:central}
    The \sdom problem parametrized by solution size admits a linear kernel on planar graphs. There exists a polynomial-time algorithms that given a planar graph $(G, k)$, either correctly reports that $(G, k)$ is a NO-instance or returns an equivalent instance $(G', k)$ such that XXX.
\end{restatable}

 \dom problem and \tdom problem, both already 

\chapter{Preliminaries}
We start by recapping some basic notation in Graph Theory and Parametrized Complexity. 

%% TODO State Draft
Continuing an intensive study of parametrized complexity of that problem. 

\section{Graph Theory}
We quickly state the following definitions given by {\cite[p.~xxx]{diestel10}}.

%% TODO Path, Subgraph, Induced Subgraph, Copnnected 
\begin{definition}[Graph]
A graph is a pair $G = (V, E)$ of two sets where $V$ denotes the vertices and $E \subseteq V \times C$ the edges of the graph.  A vertex $v \in V$ is incident with an edge $e \in E$ if $v \in e$. Two vertices $x, y$ are adjacent, or neighbours, if $\{x,y \} \in E$.
% Size of the Graph?
\end{definition}

\begin{definition}[Special Graph Notations {\cite[p.~27]{diestel10}}]
    A simple Graph

    A directed Graph is a graph

    A Multi Graph

    A Planar Graph
\end{definition}

\begin{definition}[Adjacent Vertices]
\end{definition}

\begin{definition}[Closed and Open Neighborhoods of Vertices]
+ Sets
\end{definition}

\begin{definition}[Induced Subgraph]
    asd
\end{definition}

\subsection*{Special Graph Classes}
We call the class of graphs without any special restrictions "General Graphs".

\begin{definition}[r-partite Graphs]
    Let $r \geq 2$ be an integer. A Graph $G = (V,E)$ is called "r-partite" if V admits a parititon into r classes such that every edge has its ends in different classes: Vertices in the same partition class must not be adjacent. 
    
    For the case $r = 2$ we say that the G is "bipartite" 
%      %        TODO mage of a bipartite Graph

\end{definition}

\begin{definition}[Chordal Graphs]
    
\end{definition}

\begin{definition}[Split Graphs]
    
\end{definition}

% Independent Set

\section{Parametrized Complexity}

\subsection{Fixed Parameter Tractability}
\paragraph{Fixed Parameter Intractability: The \hmath $W$ Hierarchy}
\subsection{Kernelization}


\chapter{On Parametrized Dominating Set}
\section{Semitotal Domination}

\sdom

Definition, dominating number

\subsection*{Complexity Status of \sdom}

\section{\hmath $w[i]$-Intractibility}

Now some  w[i] hard classes. 

\subsection{Warm-Up: \hmath $W[2]$-hard on General Graphs}

% TODO can we conclude anything for AT Free Graphs?
%% TODO Extend to r-partite
\subsection{\hmath $W[2]$-hard on Bipartite Graphs}

As any \bg with bipartition can be split further into \rpg this results also implies the \wone-hardness of \rpg

\subsection{\hmath $W[2]$-hard on Chordal Graphs}

\subsection{\hmath $W[2]$-hard on Split Graphs}

\chapter{A Linear Kernel for Planar Semitotal Domination}

TODO Alber et. al, Total Domination. 

\section{The Main Idea and The Big Picture}



\section{Definitions}

\begin{definition} \label{def:nv}
    Let \G be a graph and let $v \in V$. We denote by $N(v) = \{u \in V : \{u,v\} \in E \}$ the neighborhood of $v$. We split $N(v)$ into three subsets:
    \begin{align}
    N_1(v) &= \{u \in N(v) : N(u) \setminus (N(v)\cup \{v\}) \neq \emptyset \} \\
    N_2(v) &= \{u \in N(v)\setminus N_1(v) : N(u) \cap N_1(v) \neq \emptyset \} \\
    N_3(v) &= N(v) \setminus (N_1(v) \cup N_2(v))
    \end{align}
    In order to inhance future readability, we add some syntactical sugar.  For $i,j \in [1,3]$, we denote $N_{i,j} (v) = N_i(v) \cup N_j(v)$.
\end{definition}
    
\begin{definition}
    Let \G be a graph and $v,w \in V$. We denote by $N(v,w) = N(v) \cup N(w)$ the neighborhood of the pair $v,w$. We split $N(v,w)$ into three subsets:
    \begin{align}
    N_1(v,w) &= \{u \in N(v,w) \mid N(u) \setminus (N(v,w)\cup \{v,w\}) \neq \emptyset \} \\
    N_2(v,w) &= \{u \in N(v,w)\setminus N_1(v,w) \mid N(u) \cap N_1(v,w) \neq \emptyset \}\\
    N_3(v,w) &=  N(v,w) \setminus (N_1(v,w) \cup N_2(v,w))
    \end{align}
    Again, for $i,j \in [1,3]$, we denote $N_{i,j}(v,w) = N_i(v,w) \cup N_j(v,w)$.
\end{definition}

\section{Deducing Reduction Rules}

\subsection{Reduction Rule I: Getting Rid of unneccessary  $N_3(v)$ vertices}

\begin{rgl}\label{rgl:rone}
    Let \G be a graph and let $v \in V$. If $\abs{\Nthreev} \geq 1$:

    \begin{itemize}
        \item remove $\Nthreev$ from G, 
        \item add a vertex $v'$ and an edge $\{v, v'\}$
    \end{itemize}
     
\end{rgl}
\begin{lemma}
    Let \G be a a graph and let $v \in V$. If $G'$ is the graph obtained by applying \cref{rgl:rone}   on $V$, then G has SDS of size k if and only if G' has one.
\end{lemma}
\begin{proof}
   This will be the proof for this lemma X 
\end{proof}

\subsection{Reduction Rule II: Shrinking the Size of a Region}

% TODO: Force connectivity, improve kerne bound

...
% TODO: Reformulate
Following the approach from [STAU] which relies on the technique first introduced by Alber et al we try to reduce the neighborhood for two given vertices $v$ and $w$

This observation gives motivation to define the following sets:

\begin{align}
    \Dvw &= \{ \tilde D \subseteq N_{2,3}(v,w)            \mid N_3(v,w) \subseteq \bigcup_{v \in \tilde D} N(v),\ |\tilde D| \leq 3                  \}\\
    \Dv  &= \{ \tilde D \subseteq N_{2,3}(v,w) \cup \{v\} \mid N_3(v,w) \subseteq \bigcup_{v \in \tilde D} N(v),\ |\tilde D| \leq 3,\ v \in \tilde D \}\\
    \Dw  &= \{ \tilde D \subseteq N_{2,3}(v,w) \cup \{w\} \mid N_3(v,w) \subseteq \bigcup_{v \in \tilde D} N(v),\ |\tilde D| \leq 3,\ w \in \tilde D \}
    \end{align}

    %TODO Explain that more in detail. + Examples.
    Note, that 
\begin{rgl}\label{rgl:rtwo}
    Let \G be a graph and two distinct $v,w \in V$. If $\Dvw = \emptyset$ we apply the following:
    \begin{caseof}
        \case{if $\Dv =  \emptyset$ and $D_w = \emptyset$}

        \begin{itemize}
            \item Remove $N_{2,3}(v,w)$
            \item Add vertices $v'$ and $w'$ and two edges $\{v, v'\}$ and $\{w, w'\}$
            \item If there was a common neighbor of $v$ and $w$ in $N_{2,3}(v,w)$ add another vertex $y$ and two connecting edges  $\{v, y\}$ and $\{y, w\}$
        \end{itemize}
        \case{if $\Dv \neq  \emptyset$ and $D_w \neq \emptyset$}asdasd
        \case{if $\Dv \neq  \emptyset$ and $D_w = \emptyset$}asd
        \case{if $\Dv =  \emptyset$ and $D_w \neq \emptyset$} This case is symmetrical to \textbf{Case 3}. 
    \end{caseof}
\end{rgl}

Before proofing the correctness of this reduction we will give some intuition behind these rules.

\begin{lemma}
    Let \G be a plane graph, $v, w \in V$ and \GB be the graph obtained after application of \cref{rgl:rtwo} on the pair $\{v, w\}$. Then G has SDS of size k if and only if G' has SDS of size k.
\end{lemma}

\subsection{Reduction Rule III: Shrinking Simple Regions}
\begin{rgl}\label{rgl:rthree}
    Let \G be a plane graph, $v, w \ in V$ and $R$ be a simple region between $v$ and $w$. If $\abs{V(R) \setminus \{v, w\}} \geq 7$

    % TODO N_3 star?
    \begin{itemize}
        \item Remove $N_3*(v,w)$
        \item Add two vertices $h_1$ and $h_2$ and four edges $\{v, h_1\}$, $\{v, H_2\}$, $\{w, h_1\}$ and $\{w, h_2\}$
    \end{itemize}
\end{rgl}
\begin{lemma}[Correctness of \cref{rgl:rthree}]

    Let \G be a plane graph, $v, w \in V$ and \GB be the graph obtained after application of \cref{rgl:rthree} on the pair $\{v, w\}$. Then G has SDS of size k if and only if G' has SDS of size k.
\end{lemma}

The application of \cref{rgl:rthree} gives us a bound on the number of vertices inside a \sr. 
\begin{corollary}
    Let \G be a graph, $v, w\in V$ and R a \sr~ between $v$ and $w$. If \cref{rgl:three} has been applied, the simply region has size at most XXXX.
\end{corollary}

\subsection{Computing Maximal Simple Regions between two vertices}

For the sake of completeness, we state an algorithm how a maximal \sr between two vertices $v,w \in V$ can be computed in time $\mathcal{O}(d(v) + d(w))$:

%\begin{lemma}[Reduced Plane Graph under R2]
%    
%\end{lemma}


%\begin{lemma}[Given size of N1 and N2 regions] 
%    
%\end{lemma}

\section{Bounding the Size of the Kernel}


%TODO: Can I "induced" in this case is undefined
\begin{lemma}
   Given a plane Graph $G = (V,E)$ reduced under R2 and a region R(v, w), if $\Dv \neq $ (resp. $\Dw \neq \emptyset$), $N_3(v,w) \cap V(R)$ can be covered by: 
   \begin{itemize}
    \item $11$ \sr if $\Dw \neq \emptyset$, 
    \item $14$ \sr if $N_{2,3}(v) \cap N_3(v,w) = \emptyset$
   \end{itemize}
\end{lemma}

\begin{proof} 
\end{proof}

\begin{lemma}[\#Vertices inside a Region after \cref{rgl:rone,rgl:rtwo,rgl:three}]
    Let \G be a plane graph reduced under \cref{rgl:rone,rgl:rtwo,rgl:rthree}. Furthermore, let $D$ be a SDS of G and let $v,w \in D$. Then a region R between $v$ and $w$ has size at most xxx.
\end{lemma}
\begin{proof} 
\end{proof}

\begin{lemma}[\#Vertices outside a Region]
    
\end{lemma}
\begin{proof} 
\end{proof}

% TODO: CITE ALBER AND STAU
\begin{lemma}[Number of Regions in a Maximum Region Decomposition]
    
\end{lemma}
\begin{proof} 
\end{proof}


\begin{lemma}[Running Time of Reduction Procedure]
    
\end{lemma}
\begin{proof} 
\end{proof}

We now have all the weapons set up to proof \cref{thm:central}: 

\centraltheo*
\begin{proof}
    XXXXXXsdfjöakfjölajdflöajsfölajsöflkjaslödjaöslf jöalkdjf aölsfjaöslfjöalsfjaölsdföalkföalksdjföaksdföalksdföalksföalksdfaksdf ajkf öaklsdf öalkdf öal    lökjadf
\end{proof}

\chapter{Open Questions and Further Research}

* Chordal Bipartite Grap hs a very interesting case.
* Improve the Kernel Bound