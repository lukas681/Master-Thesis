\chapter{Introduction}

\subsection*{Our contributions}

\chapter{Preliminaries}
We start by recapping some basic notation in Graph Theory and Parametrized Complexity. 

%% TODO State Draft
Continuing an intensive study of parametrized complexity of that problem. 

\section{Graph Theory}
We quickly state the following definitions given by {\cite[p.~xxx]{diestel10}}.

%% TODO Path, Subgraph, Induced Subgraph, Copnnected 
\begin{definition}[Graph]
A graph is a pair $G = (V, E)$ of two sets where $V$ denotes the vertices and $E \subseteq V \times C$ the edges of the graph.  A vertex $v \in V$ is incident with an edge $e \in E$ if $v \in e$. Two vertices $x, y$ are adjacent, or neighbours, if $\{x,y \} \in E$.
% Size of the Graph?
\end{definition}

\begin{definition}[Special Graph Notations {\cite[p.~27]{diestel10}}]
    A simple Graph

    A directed Graph is a graph

    A Multi Graph

    A Planar Graph
\end{definition}

\begin{definition}[Adjacent Vertices]
\end{definition}

\begin{definition}[Closed and Open Neighborhoods of Vertices]
+ Sets
\end{definition}

\begin{definition}[Induced Subgraph]
    asd
\end{definition}

\subsection*{Special Graph Classes}
We call the class of graphs without any special restrictions "General Graphs".

\begin{definition}[r-partite Graphs]
    Let $r \geq 2$ be an integer. A Graph $G = (V,E)$ is called "r-partite" if V admits a parititon into r classes such that every edge has its ends in different classes: Vertices in the same partition class must not be adjacent. 
    
    For the case $r = 2$ we say that the G is "bipartite" 
%      %        TODO mage of a bipartite Graph

\end{definition}

\begin{definition}[Chordal Graphs]
    
\end{definition}

\begin{definition}[Split Graphs]
    
\end{definition}

% Independent Set

\section{Parametrized Complexity}

\subsection{Fixed Parameter Tractability}
\subsection{Fixed Parameter Intractability: The \hmath $W$ Hierarchy}
\subsection{Kernelization}


\chapter{On Parametrized Dominating Set}
\section{Semitotal Domination}

Definition, dominating number

\subsection*{A Survey}

\section{\hmath $w[i]$-Intractibility}

Now some  w[i] hard classes. 

\subsection{Warm-Up: \hmath $W[2]$-hard on General Graphs}

% TODO can we conclude anything for AT Free Graphs?
%% TODO Extend to r-partite
\subsection{\hmath $W[2]$-hard on Bipartite Graphs}

\subsection{\hmath $W[2]$-hard on Chordal Graphs}

\subsection{\hmath $W[2]$-hard on Split Graphs}

\chapter{A Linear Kernel for Planar Semitotal Domination}

TODO Alber et. al, Total Domination. 

\section{The Main Idea and The Big Picture}
\section{Necessary Definitions}

\section{Deducing Reduction Rules}
\subsection{Reduction Rule I (R1): Getting Rid of unneccessary $N_3(v)$ vertices}

\begin{lemma}[Correctness of the Reduction]
    
\end{lemma}

\subsection{Reduction Rule II (R2): Shrinking the Size of a Region}
\begin{lemma}[Correctness of the Reduction]
    
\end{lemma}

\subsection{ Reduction Rule III (R3): Shrinking the Size of Simple Regions}
\begin{lemma}[Correctness of the Reduction]
    
\end{lemma}

%TODO Cite Stau, Ignasi
\begin{lemma}[Reduced Plane Graph under R2]
    
\end{lemma}


\begin{lemma}[Given size of N1 and N2 regions] 
    
\end{lemma}

\section{Bounding the Size of the Kernel}


%TODO: Can I "induced" in this case is undefined
\begin{lemma}
   Given a plane Graph $G = (V,E)$ reduced under R2 and a region R(v, w), if $\Dv \neq \emptyset $ (resp. $\Dw \neq \emptyset$), $N_3(v,w) \cap V(R)$ can be covered by: 
   \begin{itemize}
    \item $11$ simple regions if $\Dw \neq \o$, 
    \item $14$ simple regions if $N_{2,3}(v) \cap N_3(v,w) = \o$
   \end{itemize}
\end{lemma}

\begin{proof} 
\end{proof}

\begin{lemma}[Number of Vertices inside a Region ]
    
\end{lemma}
\begin{proof} 
\end{proof}

\begin{lemma}[Number of Vertices outside a Region]
    
\end{lemma}
\begin{proof} 
\end{proof}

\begin{lemma}[Number of Regions in a Maximum Region Decomposition]
    
\end{lemma}
\begin{proof} 
\end{proof}

We now have all the tools ready to proof the central theorem of this section: 

\begin{lemma}[Running Time of Reduction Procedure]
    
\end{lemma}
\begin{proof} 
\end{proof}

\begin{theorem}
    Semitotal Dominating Set has a linear kernel of size XXX on planar graphs.
\end{theorem}
\begin{proof} 
\end{proof}

\subsection*{NP Hardness on General Graphs}

\chapter{Open Questions and Further Research}

Chordal Bipartite Graphs a very interesting case.