\chapter{Introduction}


Parametrized Complexity emerging branch. Books about that

Semitotal domination introduced by 
% TODO: Idea for a nice introduction 
\section{Content of the thesis}

In this thesis we continue the systematic analysis of the \sdom problem by focusing on the parametrized complexity of the problem. 

Although the problem already had a lot of attention regarding classical complexity (CITE), only few results are currently known for the parametrized variant. 

As far as we have seen, even the w-hardness of the general case has not been explicitely been proofen in the literature. 

In this thesis we continue the journey towards a systematic analysis by stating some hardness results for specific graph classes for the problem.

\paragraph{Our contributions}
% TODO Better: 

Our main contributations consist of first showing the $w[2]$-hardness of \sdom for XXXX graphs.

\noindent As the \dom problem and the \tdom problem both admit a linear kernel for planar graphs, it is interesting to analyse wether this results also holds for the \sdom problem which lays in between these two. 
%TODO by relxing the witness of these two provlemsproblems.

Having these kernels also for other variants like \eddom, \efdom, \cdom, \rbdom lent us a great confidence that the result will also work for \sdom on planar graphs.

%% TODO Find more.

Following the approach from ... which alraedy relies on the technique given in, we give some simple data reduction rules for \sdom on planar graphs leading to a linear kernel. More precisely, we are going to proof the following central theorem of this thesis:

With some  modifications we were able to transfer the approach given by Garnero and Stau in \cite{Garnero2018} to the \sdom problem.

\begin{restatable}[]{theorem}{centraltheo}\label{thm:central}
    The \sdom problem parametrized by solution size admits a linear kernel on planar graphs. There exists a polynomial-time algorithms that given a planar graph $(G, k)$, either correctly reports that $(G, k)$ is a NO-instance or returns an equivalent instance $(G', k)$ such that XXX.
\end{restatable}

 \dom problem and \tdom problem, both already 

\chapter{Preliminaries}
We start by recapping some basic notation in Graph Theory and Parametrized Complexity. 

%% TODO State Draft
Continuing an intensive study of parametrized complexity of that problem. 

\section{Graph Theory}
We quickly state the following definitions given by {\cite[p.~xxx]{diestel10}}.

%% TODO Path, Subgraph, Induced Subgraph, Copnnected 
\begin{definition}[Graph]
A graph is a pair $G = (V, E)$ of two sets where $V$ denotes the vertices and $E \subseteq V \times C$ the edges of the graph.  A vertex $v \in V$ is incident with an edge $e \in E$ if $v \in e$. Two vertices $x, y$ are adjacent, or neighbours, if $\{x,y \} \in E$.
% Size of the Graph?
\end{definition}

\begin{definition}[Special Graph Notations {\cite[p.~27]{diestel10}}]
    A simple Graph

    A directed Graph is a graph

    A Multi Graph

    A Planar Graph
\end{definition}

\begin{definition}[Adjacent Vertices]
\end{definition}

\begin{definition}[Closed and Open Neighborhoods of Vertices]
+ Sets
\end{definition}

\begin{definition}[Induced Subgraph]
    asd
\end{definition}

\begin{definition}[Isomorphic Graph]
    asd
\end{definition}

\subsection*{Special Graph Classes}
We call the class of graphs without any special restrictions "General Graphs".

\begin{definition}[r-partite Graphs]
    Let $r \geq 2$ be an integer. A Graph $G = (V,E)$ is called "r-partite" if V admits a parititon into r classes such that every edge has its ends in different classes: Vertices in the same partition class must not be adjacent. 
    
    For the case $r = 2$ we say that the G is "bipartite" 
%      %        TODO mage of a bipartite Graph

\end{definition}

\begin{definition}[Chordal Graphs]
    
\end{definition}

\begin{definition}[Split Graphs]
    
\end{definition}

% Independent Set

\section{Parametrized Complexity}

\subsection{Fixed Parameter Tractability}
\paragraph{Fixed Parameter Intractability: The \hmath $W$ Hierarchy}
\subsection{Kernelization}


\chapter{On Parametrized Semitotal Domination}
\section{Semitotal Domination}

\sdom

Definition, dominating number

\subsection*{Complexity Status of \sdom}

\section{\hmath $w[i]$-Intractibility}

Now some  w[i] hard classes. 

\subsection{Warm-Up: \hmath $W[2]$-hard on General Graphs}

% TODO can we conclude anything for AT Free Graphs?
%% TODO Extend to r-partite

As any \bg with bipartition can be split further into \rpg this results also implies the \wone-hardness of \rpg

\subsection{\hmath $W[2]$-hard on Bipartite Graphs}

\begin{definition}[Bipartite Graph, {\cite[p.5]{Bondy2008}}]
    
A \textit{\bg} is a Graph G whose vertex set can be partitioned into two subsets X and Y, so that each edge has one end in X and one end in Y. Such a partition (X,Y) is called a \textit{bipartition} of G.

\end{definition}

\begin{figure}[htb]
    \centering
\definecolor{myblue}{RGB}{80,80,160}
\definecolor{mygreen}{RGB}{80,160,80}

\resizebox{0.9\textwidth}{!}{
\begin{tikzpicture}[thick,
        every node/.style={draw,circle},
        fsnode/.style={fill=myblue},
        ssnode/.style={fill=mygreen},
        every fit/.style={ellipse,draw,inner sep=-2pt,text width=2.2cm},
        -,shorten >= 3pt,shorten <= 3pt
    ]
    % the vertices of U
    \begin{scope}[xshift=2cm,start chain=going below, node distance=7mm]
        \foreach \i in {1,2,...,5}
        \node[fsnode,on chain] (f\i) [label=above left: $x_\i$] {};
    \end{scope}

    \begin{scope}[start chain=going below, node distance=7mm]
        \foreach \i [count=\j] in {6,7,...,10}
        \node[ssnode,on chain] (f\i) [label=above left: $x'_{\j}$] {};
    \end{scope}

    % the vertices of V
    \begin{scope}[xshift=5cm,yshift=-0.5cm,start chain=going below, node distance=7mm]
        \foreach \i [count=\j] in {11,12,...,14}
        \node[ssnode,on chain] (f\i) [label=above right: $y_{\j}$] {};
    \end{scope}
    
    \begin{scope}[xshift=7cm,yshift=-0.5cm,start chain=going below, node distance=7mm]
        \foreach \i [count=\j] in {15,16,...,18}
        \node[fsnode,on chain] (f\i) [label=above right: $y'_{\j}$] {};
    \end{scope}

    \node [fsnode, left=of f8, xshift=-1cm] (nx) [label=above left: $d_1$]{};
    \node [ssnode, right=of nx, xshift=11cm] (ny) [label=above right: $d_2$]{};

    \node [ssnode, left=of nx] (nxx) [label=left: $u_1$]{};
    \node [fsnode, right=of ny] (nyy) [label=right: $u_2$]{};

    % the set U
    \node [myblue,fill=aqua, opacity=0.1,fit=(f1) (f5),label=above:$X$] {};
    \node [mygreen,fill=applegreen, opacity=0.1,fit=(f11) (f14),label=above:$Y$] {};

    \node [mygreen,fill=applegreen, opacity=0.1, fit=(f6) (f10),label=above:$ $] {};
    \node [myblue,fill=aqua, opacity=0.1, fit=(f18) (f15),label=above:$ $] {};
     
    % the set V
    % \node [mygreen,fit=(s6) (s9),label=above:$V$] {};

    % the edges
    \draw (f1) -- (f11);
    \draw (f1) -- (f12);
    \draw (f1) -- (f14);
    \draw (f2) -- (f14);
    \draw (f3) -- (f13);
    \draw (f3) -- (f11);
    \draw (f5) -- (f12);
    \draw (f4) -- (f14);
    
    \foreach \i in {6,7,...,10}
    \draw (nx) -- (f\i);

    \foreach \i in {15,16,...,18}
    \draw (ny) -- (f\i);

    \draw (ny) -- (nyy);
    \draw (nx) -- (nxx);

    % The doubled edges
    \draw (f6) -- (f1);
    \draw (f7) -- (f2);
    \draw (f8) -- (f3);
    \draw (f9) -- (f4);
    \draw (f10) -- (f5);

    \draw (f11) -- (f15);
    \draw (f12) -- (f16);
    \draw (f13) -- (f17);
    \draw (f14) -- (f18);

\end{tikzpicture}
}
\caption{Constructing G' from a bipartite Graph G by duplicating the vertices and adding a dominating tail}
\end{figure}


\begin{theorem}
    Semitotal Dominating Set is $\omega[2]$ hard for bipartite Graphs
\end{theorem}

\begin{proof}
    Given a bipartite Graph $G = ( \{X \cup Y\}, E)$, we construct a bipartite Graph G' in the following way:
    \begin{enumerate}
        \item For each vertex $x_i \in X$ we add a new vertex $x_i'$  and an edge $(x_i, x_i')$ in between.
        \item For each vertex $y_j \in Y$ we add a new vertex $y_j'$ and an edge $(y_j, y_j')$ in between.
        \item We add two $P_1$, namely $(u_1, d_1)$ and $(u_2, d_2)$, and connect them with all $(d_1, x_i')$ and $(d_2, y_j')$ respectively.
    \end{enumerate}
    \paragraph*{Observation:} G' is clearly bipartite as all $y'_j$ and $x'_i$ form again an Independent Set. Setting  $X' = X \cup \{u_2\} \cup \bigcup y'_i$ and $Y' = Y \cup \{u_1\} \cup \bigcup {x'_i}$ form the partitions of bipartite G'.

    \begin{corollary} G has a Dominating Set of size k iff G has a Semitotal Dominating Set of size $k' = k + 2$
    \end{corollary} 
    $\Rightarrow$: Asume there exists a Dominating Set D in G with size k. $DS = D\cup \{d_1,d_2\}$ is a Semitotal Dominating Set in $G'$ with size $k' = k+2 $ , because $d_1$ dominates $u_1$ and all $x'_i$; $d_2$ dominates $u_2$ and all $y'_i$. Hence, it is a Semitotal Dominating Set, because $\forall v \in (D \cap X): d(v, d_1) = 2$ and $\forall v \in (D \cap Y): d(v, d_2) = 2$

    $\Leftarrow$: On the contrary, asume any Semitotal Dominating Set $SD$ in $G'$ with size $k'$. WLOG we can asume that $u_1, u_2 \notin DS$. 
    
    Our construction forces $d_1$, $d_2 \in DS$. Because all $x'_i$ are only important in dominating $x_i$ ($y'_i$ for $y_i$ resp.) as $d_1, d_2 \in DS$. If $x'_i \in DS$ we simply exchange it with $x_i$ (for $y'_i$ and $y_i$ respectively) in our DS keeping the size of the dominating set. $D = DS \setminus \{ d_1,d_2\}$ give us a Dominating Set in G with size $ k = k' - 2$

    As G' can be constructed in $\mathcal{O}(n)$ and parameter k is only blown up by a constant, this reduction is a FPT reduction. As Dominating Set is $w[2]$ hard for bipartite Graphs\footnote{Citation needed!} so is Semitotal Dominating Set.
\end{proof}


\subsection{\hmath $W[2]$-hard on Chordal Graphs}

\subsection{\hmath $W[2]$-hard on Split Graphs}

\chapter{A Linear Kernel for Planar Semitotal Domination}

\epigraph{\itshape The best way to explain it is to do it.}{Lewis Caroll, \textit{Alice in Wonderland}}

We are now building up towards a linear kernel for the \sdom problem. In order to achieve this, we will first split up the neighborhood of one vertice and a pair of vertices into three distinct subsets, give some nice properties on them and then state the corresponding reduction rules.  

% TODO: Reword

%TODO: SIZE of instance: Number of vertices.

But first, we would like to define what we consider to be a \textit{reduced} graph. 

\begin{definition}[Reduced Graph {\cite[p. 13]{Garnero2018}} and \cite{Garnero2017}]
    A Graph G is reduced under a set of rules if either none of these rules can be applied to G or the application of any of them creates a graph isomorphic to G.
\end{definition}

In our case, we say G is reduced if none of the \cref{rgl:rone,rgl:rtwo,rgl:rthree} are modifying G any more.

This differs from the definition usually giving in literature where a graph G is \textit{reduced} under a set of reduction rules, if none of them can be applied to G anymore (Compare e.g. \cite{Fomin2019}). The reason is that we are giving reduction rules (see \cref{rgl:rone} or \cref{rgl:rtwo}) that could be applied \textit{ad infinitum} sending us into an endless loop without ever changing G any more. Our definition guarantees termination in that case.

From an algorithmic point of view, all our given reduction rules are local and only concern the neighborhood of at most two vertices and replace them partially with gadgets of constant size. Now checking wether a graph after applying the rule has beein changed can be trivially be accomplished in constant time.

\section{The Main Idea and The Big Picture}

[TODO SUM UP THE STRATEGY]


\section{Definitions}

We are now going to split up the neighborhood of a single vertex into three (disjoint) subsets.


\begin{definition} \label{def:nv}
    Let \G be a graph and let $v \in V$. We denote by $N(v) = \{u \in V : \{u,v\} \in E \}$ the neighborhood of $v$. We split $N(v)$ into three subsets:
    \begin{align}
    N_1(v) &= \{u \in N(v) : N(u) \setminus (N(v)\cup \{v\}) \neq \emptyset \} \\
    N_2(v) &= \{u \in N(v)\setminus N_1(v) : N(u) \cap N_1(v) \neq \emptyset \} \\
    N_3(v) &= N(v) \setminus (N_1(v) \cup N_2(v))
    \end{align}
    In order to inhance future readability, we add some syntactical sugar.  For $i,j \in [1,3]$, we denote $N_{i,j} (v) = N_i(v) \cup N_j(v)$.
\end{definition}

\noindent \textbf{$N_1(v)$} are all neighbors of $v$ which have at least one adjacent vertex that is outside of $N(v)$.

\noindent \textbf{$N_2(v)$} are all neighbors of $v$ which have at least one adjacent vertex that is outside of $N(v)$.

\noindent \textbf{$N_3(v)$} are all neighbors of $v$ which have at least one adjacent vertex that is outside of 

\begin{tikzpicture}[scale=1.6, thick]
	\begin{pgfonlayer}{nodelayer}
		\node [style=BLACK, label={above:$v$}] (0) at (1, 2) {};
		\node [style=NONE] (1) at (4.25, 2.75) {};
		\node [style=NONE] (2) at (4.25, 0) {};
		\node [style=NTHR] (3) at (5.25, 1.75) {};
		\node [style=NTHR] (4) at (5.75, 2.75) {};
		\node [style=NTHR] (5) at (5.25, 3.75) {};
		\node [style=NTHR] (6) at (5.25, 1) {};
		\node [style=NTHR] (7) at (5.75, 0) {};
		\node [style=NTHR] (8) at (5.25, -1) {};
		\node [style=NTHR] (9) at (2, 4) {};
		\node [style=NTWO] (10) at (1.25, 0.25) {};
		\node [style=NTWO] (11) at (3, 3.75) {};
		\node [style=NTWO] (12) at (0.25, -0.5) {};
		\node [style=NTHR] (14) at (-0.5, 3.75) {};
		\node [style=NTHR] (15) at (-1.5, 2.25) {};
	\end{pgfonlayer}
	\begin{pgfonlayer}{edgelayer}
		\draw (0) to (1);
		\draw (0) to (2);
		\draw (4.center) to (1);
		\draw (3.center) to (1);
		\draw (1) to (5.center);
		\draw (6.center) to (2);
		\draw (7.center) to (2);
		\draw (8.center) to (2);
		\draw (1) to (2);
		\draw (9) to (0);
		\draw (0) to (11);
		\draw (11) to (1);
		\draw (10) to (0);
		\draw (10) to (2);
		\draw (0) to (12);
		\draw (12) to (2);
		\draw (0) to (15);
		\draw (12) to (15);
		\draw (14) to (15);
		\draw (14) to (0);
	\end{pgfonlayer}
\end{tikzpicture}


Note that from a semitotal dominating perspective, they are only useful as witnesses for $v$, because it follows from definition that for a $z \in N_3(v)$ it holds that  $N(z) \subseteq N(v)$ and thus, we would always prefer $v$ as a dominating vertex instead of $z$. We are using thise observation in \cref{rgl:rone} where we shrink $\abs{N_3(v)} \leq 1$ 




\begin{definition}
    Let \G be a graph and $v,w \in V$. We denote by $N(v,w) = N(v) \cup N(w)$ the neighborhood of the pair $v,w$. We split $N(v,w)$ into three subsets:
    \begin{align}
    N_1(v,w) &= \{u \in N(v,w) \mid N(u) \setminus (N(v,w)\cup \{v,w\}) \neq \emptyset \} \\
    N_2(v,w) &= \{u \in N(v,w)\setminus N_1(v,w) \mid N(u) \cap N_1(v,w) \neq \emptyset \}\\
    N_3(v,w) &=  N(v,w) \setminus (N_1(v,w) \cup N_2(v,w))
    \end{align}
    Again, for $i,j \in [1,3]$, we denote $N_{i,j}(v,w) = N_i(v,w) \cup N_j(v,w)$.
\end{definition}

\section{Deducing Reduction Rules}

Following the approach by \cite{Garnero2014}, we are now stating reduction rules that after exhaustive application will expose a linear kernel. 

\subsection{Reduction Rule I: Getting Rid of unneccessary  $N_3(v)$ vertices}

\begin{rgl}\label{rgl:rone}
    Let \G be a graph and let $v \in V$. If $\abs{\Nthreev} \geq 1$:

    \begin{itemize}
        \item remove $\Nthreev$ from G, 
        \item add a vertex $v'$ and an edge $\{v, v'\}$
    \end{itemize}
     
\end{rgl}
\begin{lemma}
    Let \G be a a graph and let $v \in V$. If $G'$ is the graph obtained by applying \cref{rgl:rone}   on $V$, then G has SDS of size k if and only if G' has one.
\end{lemma}
\begin{proof}
   This will be the proof for this lemma X 
\end{proof}

Note: If we would not consider reduced graph to be ...

\subsection{Reduction Rule II: Shrinking the Size of a Region}

% TODO: Force connectivity, improve kerne bound

% TODO: Reformulate
Extending the approach for a linear kernel for \dom proposed by Alber et al. in \cite{Alber2004}, Garnero and Stau transferred these results in \cite{Garnero2018} to the \tdom problem. 

%TODO: Better:
Their idea was to relax the reduction rules in such a way that the witness properties for total domination are being preserved.

% TODO Should we discuss that?
Following their approach in one of the first verions of \cite{Garnero2014}, we stating reduction rules that. Interestingly, the reduction rules given in the latest version of this paper was not transferable to \sdom, but an older version giving slightly easier reduction rules could be adjusted to our problem.

which relies on the technique first introduced by Alber et al we try to reduce the neighborhood for two given vertices $v$ and $w$

This observation gives motivation to define the following sets:

\begin{align}
    \Dvw &= \{ \tilde D \subseteq N_{2,3}(v,w)            \mid N_3(v,w) \subseteq \bigcup_{v \in \tilde D} N(v),\ |\tilde D| \leq 3                  \}\\
    \Dv  &= \{ \tilde D \subseteq N_{2,3}(v,w) \cup \{v\} \mid N_3(v,w) \subseteq \bigcup_{v \in \tilde D} N(v),\ |\tilde D| \leq 3,\ v \in \tilde D \}\\
    \Dw  &= \{ \tilde D \subseteq N_{2,3}(v,w) \cup \{w\} \mid N_3(v,w) \subseteq \bigcup_{v \in \tilde D} N(v),\ |\tilde D| \leq 3,\ w \in \tilde D \}
    \end{align}

    %TODO Explain that more in detail. + Examples.
    Note, that 
\begin{rgl}\label{rgl:rtwo}
    Let \G be a graph and two distinct $v,w \in V$. If $\Dvw = \emptyset$ we apply the following:
    \begin{caseof}
        \case{if $\Dv =  \emptyset$ and $D_w = \emptyset$}

        \begin{itemize}
            \item Remove $N_{2,3}(v,w)$
            \item Add vertices $v'$ and $w'$ and two edges $\{v, v'\}$ and $\{w, w'\}$
            \item If there was a common neighbor of $v$ and $w$ in $N_{2,3}(v,w)$ add another vertex $y$ and two connecting edges  $\{v, y\}$ and $\{y, w\}$
        \end{itemize}
        \case{if $\Dv \neq  \emptyset$ and $D_w \neq \emptyset$}
        % TODO: If it can not be applied any more. Using R3 to reduce this      
        Do nothing\footnote{Originally, reduce Simple Regions [STAU]}

        \case{if $\Dv \neq  \emptyset$ and $D_w = \emptyset$}

        \begin{itemize}
            \item Remove $N_{2,3}(v) \Cap N_3(v,w)$
            \item Add $\{v, v'\}$ 
        \end{itemize}

        \case{if $\Dv =  \emptyset$ and $D_w \neq \emptyset$} This case is symmetrical to \textbf{Case 3}. 
    \end{caseof}
\end{rgl}

Before proofing the correctness of this reduction we will give some intuition behind these rules.

\begin{lemma}
    Let \G be a plane graph, $v, w \in V$ and \GB be the graph obtained after application of \cref{rgl:rtwo} on the pair $\{v, w\}$. Then G has SDS of size k if and only if G' has SDS of size k.
\end{lemma}

\subsection{Reduction Rule III: Shrinking Simple Regions}
\begin{rgl}\label{rgl:rthree}
    Let \G be a plane graph, $v, w \ in V$ and $R$ be a simple region between $v$ and $w$. If $\abs{V(R) \setminus \{v, w\}} \geq 7$

    % TODO N_3 star?
    \begin{itemize}
        \item Remove $N_3*(v,w)$
        \item Add two vertices $h_1$ and $h_2$ and four edges $\{v, h_1\}$, $\{v, H_2\}$, $\{w, h_1\}$ and $\{w, h_2\}$
    \end{itemize}
\end{rgl}
\begin{lemma}[Correctness of \cref{rgl:rthree}]

    Let \G be a plane graph, $v, w \in V$ and \GB be the graph obtained after application of \cref{rgl:rthree} on the pair $\{v, w\}$. Then G has SDS of size k if and only if G' has SDS of size k.
\end{lemma}

The application of \cref{rgl:rthree} gives us a bound on the number of vertices inside a \sr. 
\begin{corollary}
    Let \G be a graph, $v, w\in V$ and R a \sr~ between $v$ and $w$. If \cref{rgl:three} has been applied, the simply region has size at most XXXX.
\end{corollary}

\subsection{Computing Maximal Simple Regions between two vertices}

For the sake of completeness, we state an algorithm how a maximal \sr between two vertices $v,w \in V$ can be computed in time $\mathcal{O}(d(v) + d(w))$:

%\begin{lemma}[Reduced Plane Graph under R2]
%    
%\end{lemma}


%\begin{lemma}[Given size of N1 and N2 regions] 
%    
%\end{lemma}

\section{Bounding the Size of the Kernel}


%TODO: Can I "induced" in this case is undefined
\begin{lemma}
   Given a plane Graph $G = (V,E)$ reduced under R2 and a region R(v, w), if $\Dv \neq $ (resp. $\Dw \neq \emptyset$), $N_3(v,w) \cap V(R)$ can be covered by: 
   \begin{itemize}
    \item $11$ \sr if $\Dw \neq \emptyset$, 
    \item $14$ \sr if $N_{2,3}(v) \cap N_3(v,w) = \emptyset$
   \end{itemize}
\end{lemma}

\begin{proof} 
\end{proof}

\begin{lemma}[\#Vertices inside a Region after \cref{rgl:rone,rgl:rtwo,rgl:three}]
    Let \G be a plane graph reduced under \cref{rgl:rone,rgl:rtwo,rgl:rthree}. Furthermore, let $D$ be a SDS of G and let $v,w \in D$. Then a region R between $v$ and $w$ has size at most xxx.
\end{lemma}
\begin{proof} 
\end{proof}

\begin{lemma}[\#Vertices outside a Region]
    
\end{lemma}
\begin{proof} 
\end{proof}

% TODO: CITE ALBER AND STAU
\begin{lemma}[Number of Regions in a Maximum Region Decomposition]
    
\end{lemma}
\begin{proof} 
\end{proof}


\begin{lemma}[Running Time of Reduction Procedure]
    
\end{lemma}
\begin{proof} 
\end{proof}

We now have all the weapons set up to proof \cref{thm:central}: 

\centraltheo*
\begin{proof}
    XXXXXXsdfjöakfjölajdflöajsfölajsöflkjaslödjaöslf jöalkdjf aölsfjaöslfjöalsfjaölsdföalkföalksdjföaksdföalksdföalksföalksdfaksdf ajkf öaklsdf öalkdf öal    lökjadf
\end{proof}

\chapter{Open Questions and Further Research}

* Chordal Bipartite Grap hs a very interesting case.
* Improve the Kernel Bound