\chapter{Abstract}

\begin{abstract}
{\sffamily
For a graph \G , a set $D$ is called a \sdom, if $D$ is a dominating set and every vertex $v \in D$ is within distance two to another witness $v' \in D$. The \msdom problem is to find a semi-total dominating set of minimum cardinality. The semitotal domination number $\gamma_{t2}(G)$ is the minimum cardinality of a semitotal dominating set and is squeezed between the domination number $\gamma(G)$ and the total domination number $\gamma_t(G)$. Given a graph \G and a positive integer $k$, the \sdomd problem asks if $G$ has a \sdom of size at most $k$.

After the problem was introduced by Goddard, Henning and McPillan in \cite{Goddard2014}, NP-completeness was shown for general graphs \cite{Henning2019}, \emph{split graphs} \cite{Henning2019}, \emph{planar graphs} \cite{Henning2019}, \emph{chordal bipartite graphs} \cite{Henning2019}, \emph{circle graphs} \cite{Kloks2021} and \emph{subcubic line graphs of bipartite graphs} \cite{Galby2020}. On the other side, there exist polynomial-time algorithms for \emph{AT-free graphs} \cite{Kloks2021}, \emph{graphs of bounded mim-width} \cite{Galby2020}, \emph{graphs of bounded clique-width} \cite{Courcelle1990}, and \emph{interval graphs} \cite{Henning2019}.

In this thesis, we start the systematic look through the lens of \textit{parametrized complexity} by showing that \sdom is $\omega[2]$-hard for bipartite graphs and split graphs.
% TODO: What do we show for chordal graphs?
By applying the techniques proposed in \cite{Alber2004} and \cite{Garnero2018} for \dom and \tdom, we are going to construct a $\kernelsize k $ kernel for \sdom in planar graphs. This result further complements known linear kernels for other domination problems like \pcdom, \prbdom, \pefdom, \peddom, \idom and \pdirdom.

\textbf{Keywords: }Domination; Semitotal Domination; Parametrized Complexity; Planar Graphs; Linear Kernel 

}
\end{abstract}

\chapter{Zusammenfassung}

\begin{abstract}
{\sffamily
Hier kommt noch ein weiterer Abstract rein.

\textbf{Schlagworte}: Stabile Menge; Halbtotale Stabile Menge; Parametrisierte Komplexität; Plättbare Graphen; Linearer Problemkern 

}
\end{abstract}

\textit{Abstract all the way}