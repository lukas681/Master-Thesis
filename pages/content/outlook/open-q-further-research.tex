\chapter{Open Questions and Further Research}\label{ch:closing}

\vspace*{-50pt}

\begin{figure}[ht]
        \includegraphics[width=0.35\textwidth, right]{img/further-questions.png}
        \captionsetup{textformat=empty,labelformat=blank}
        \caption[Generated with Dalle-E. Knowledge Cutoff 09-2022]{Generated with Dall-E. \url{https://labs.openai.com/}. ``more ducks asking further questions and research topics''}
\end{figure}

\epigraph{\itshape ``Peter does what he usually does when he doesn’t know what to do next: he gives up.''}{Qualityland, \textit{Marc-Uwe Kling}}

This chapter discusses open problems for \sdom for further research.

\noindent \textbf{Improving Kernel~}
% Simple reembedding Procedure
%In \cref{lemma:outside} we use the fact that after \cref{rgl:rone} has been applied $N_3(v) \leq 1$ \cref{rgl:rone} and this vertex can lie in a \dreg of a fixed plane embedding. 
%The idea would be to 
%This idea has already been used in \cite[Ch. 6]{Halseth2016} and would improve the overall kernel size by one.
% Better improvement
With $\kernelsize \cdot k$, the size of our kernel is very high and the theoretical bounds are too large for practical applications. 
It would be interesting to improve them it. 
One idea would be the following:

\cref{rgl:rtwo} uses the fact that $N(v,w)$ can be dominated by at most four vertices: $v$,$w$ and two neighbors as a witness.
Observe that if $d(v,w) \leq 3$, this witness might be shared in an sds, because choosing one single witness on the path from $v$ to $w$ is sufficient to semitotally dominate $N(v,w)$.
Therefore $\Dvw$, $\Dv$ and $\Dw$ could be redefined to contain sets of size at most two (instead of three) which will improve the reduction. 
Note that in our analysis, the poles of a region $R$ in the \dreg $\mathfrak{R}$ satisfy $d(v,w) \leq 3$ by definition.
therefore, requiring $d(v,w) \leq 3$ would be ok and we can still assume that \cref{rgl:rtwo} has been applied for any $R \in \mathfrak{R}$.
$d(v,w)$ can be calculated in linear time and would not blow up our runtime.

\noindent \textbf{Experimental Results~}
According to experiments shown by Alber et al. \cite{Alber2004}, more than $79\%$ of the vertices and $88\%$ of the edges have been removed by the reduction rules from a sample set of random planar graphs with up to $4000$ vertices. 
As our theoretical bounds are quite high, it would be interesting to see how much our reduction rules remove in practice. 

\noindent \textbf{Other Open Prolems~}
The classical complexities for \textit{dually chordal} and \textit{tolerance} graphs have already been asked for by Galby et al. \cite{Galby2020} and are still open.

Furthermore, it would be interesting to complement the parameterized complexities on the hard classes we have started in this work.
open are those for \textit{circle}, \textit{chordal bipartite} and \textit{undirected path} graphs.
We think that \sdoms on \textit{chordal bipartite} graphs is at least \WONEhs-hard when parameterized by solution size, but we have been unable to prove it.
Furthermore, the reduction for \textit{circle} graphs \cite{Kloks2021} to show \NP-completeness depends on the input size, but \doms and \tdoms have already been shown to be \WONEhs-hard on \textit{circle} graphs \cite{Bousquet2012}. 
Maybe these reductions can be adjusted to show \WONEhs-hardness for \sdoms as well.
Last but not least, there exists an fpt algorithm for \doms on \textit{undirected path} graphs \cite{Figueiredo2022}. 
Does it transfer to \sdoms as well?

